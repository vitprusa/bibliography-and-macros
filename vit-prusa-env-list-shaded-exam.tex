% This will work with \usepackage{thmtools} and corresponding changes in vit-prusa-env-list.tex (comment out \newtheorem{theorem} and \newtheorem{definition}.)
% http://tex.stackexchange.com/questions/34405/shaded-theorems-thmtools-spanning-pages 

% \definecolor{mybg}{rgb}{0.9,0.9,0.9}

% \declaretheoremstyle[
% headfont=\normalfont\bfseries,
% notefont=\mdseries, notebraces={(}{)},
% bodyfont=\normalfont\itshape,
% postheadspace=0.5em,
% preheadhook={\begin{mdframed}[backgroundcolor=mybg, %
%   innertopmargin =0pt , splittopskip = \topskip, % 
%   skipbelow= 10em, skipabove=\topskip, %
%   topline=false,bottomline=false,leftline=false,rightline=false]},
% postfoothook=\end{mdframed}
% ]{mystyle}
%\declaretheorem[style=mystyle]{theorem}

%\usepackage{thmtools} Please add this to preamble!

\declaretheorem[shaded]{theorem}
\declaretheorem[shaded,sibling=theorem]{definition}
\declaretheorem[shaded,sibling=theorem]{corollary}
\declaretheorem[shaded,sibling=theorem]{lemma}
\declaretheorem[sibling=theorem]{proposition}
\declaretheorem[sibling=theorem]{example}
\declaretheorem[sibling=theorem]{exercise}
\declaretheorem[sibling=theorem]{problem}
\declaretheorem[sibling=theorem]{remark}

%\numberwithin{equation}{section}

% % Environment for solution, corresponds to the definition of "proof" environment as taken from amsbook.cls
% \makeatletter
% \newenvironment{solution}[1][\solutionname]{\par
%   \pushQED{\qed}%
%   \normalfont \topsep6\p@\@plus6\p@\relax
%   \trivlist
%   \itemindent\normalparindent
%   \item[\hskip\labelsep
%         \scshape
%     #1\@addpunct{.}]\ignorespaces
% }{%
%   \popQED\endtrivlist\@endpefalse
% }
% \providecommand{\solutionname}{Solution}
% \makeatother


% theorems, remarks, definitions, etc.
% requires amsmath, amsthm

%\newtheorem{theorem}{Theorem} % this must be commented out if we use \usepackage{thmtools}

%\theoremstyle{lemma}
%\newtheorem{lemma}[theorem]{Lemma}

%\theoremstyle{corollary}
%\newtheorem{corollary}[theorem]{Corollary}

%\theoremstyle{definition}
%\newtheorem{definition}[theorem]{Definition} % this must be commented out if we use \usepackage{thmtools}

%\theoremstyle{proposition}
%\newtheorem{proposition}[theorem]{Proposition}
%\theoremstyle{assumption}
%\newtheorem{assumption}[theorem]{Assumption} 
%\theoremstyle{condition}
%\newtheorem{condition}[theorem]{Condition} 

%\newtheorem{example}[theorem]{Example}
%\newtheorem{exercise}[theorem]{Exercise}
%\newtheorem{problem}[theorem]{Problem} 

%\theoremstyle{remark}
%\newtheorem{remark}[theorem]{Remark}
